\documentclass{resume}
\usepackage{zh_CN-Adobefonts_external} 
\usepackage{linespacing_fix}
\usepackage{cite}
\usepackage{fontawesome}
\usepackage{hyperref} 
\begin{document}
\pagenumbering{gobble}



%***"%"后面的所有内容是注释而非代码,不会输出到最后的PDF中
%***使用本模板,只需要参照输出的PDF,在本文档的相应位置做简单替换即可
%***修改之后,输出更新后的PDF,只需要点击Overleaf中的“Recompile”按钮即可
%**********************************姓名********************************************
\name{龚开宸}
%**********************************联系信息****************************************
%第一个括号里写手机号,第二个写邮箱
\contactInfo{18270359936}{gongkch2024@shanghaitech.edu.cn}
\otherInfo{性别:男}{籍贯:江西}{}{}
%**********************************其他信息****************************************
%在大括号内填写其他信息,最多填写4个,但是如果选择不填信息,
%那么大括号必须空着不写,而不能删除大括号。
%\otherInfo后面的四个大括号里的所有信息都会在一行输出
%如果想要写两行,那就用两次这个指令(\otherInfo{}{}{}{})即可
%*********************************照片**********************************************
%照片需要放到images文件夹下,名字必须是you.jpg,如果不需要照片可以不添加此行命令
%0.15的意思是,照片的宽度是页面宽度的0.15倍,调整大小,避免遮挡文字
 \yourphoto{0.12}
%**********************************正文**********************************************


%***大标题,下面有横线做分割
%***一般的标题有:教育背景,实习(项目)经历,工作经历,自我评价,求职意向,等等


%***********一行子标题**************
%***第一个大括号里的内容向左对齐,第二个大括号里的内容向右对齐
%***\textbf{}括号里的字是粗体,\textit{}括号里的字是斜体


%***********列举*********************
%***可添加多个\item,得到多个列举项,类似的也可以用\textbf{}、\textit{}做强调
\section{基本信息}
\textbf{本科 GPA: } 3.72 / 4 \\
\textbf{本科 专业排名: } 12 / 211 (5.3\%)  \\
\textbf{语言技能: } CET-4: 571 \quad CET-6: 568 \\
\begin{tabular}{@{}p{0.7\textwidth}p{0.3\textwidth}@{}}
  \textbf{本科就读院校: } 南昌大学 计算机科学与技术(\textbf{211工程, 双一流建设高校}) & \hfill \textbf{2020-2024} \\
  \textbf{研究生在读院校: } 上海科技大学 计算机科学与技术(\textbf{双一流建设高校}) & \hfill 2024-2027
\end{tabular}

% \end{itemize}

\section{项目经历}
\datedsubsection{\href{https://git.acwing.com/trace4869/xv6}{\textbf{XV6操作系统 内核增强}}}{}
\begin{itemize}[leftmargin=*]
    \item \textbf{项目概述}: 基于类 Unix 的 XV6 操作系统,深入阅读源码及官方文档,重点在性能优化和底层调试。  
    \item \textbf{系统调用实现}: 增加了 \textbf{Backtrace} 和 \textbf{Sysinfo} 等系统调用,
    其中 \textbf{Backtrace} 通过栈帧打印函数调用链,提高了调试效率;
    \textbf{Sysinfo} 可监控当前可用内存和非阻塞进程数量,从而有效监控系统运行状态。
    \item \textbf{内存分配器改进}: 重新设计 XV6 的内存分配机制,为每个进程维护独立的空闲内存链表,
    减少 \textbf{Kalloc} 中的锁竞争,提高并发性能与内存分配效率。
    \item \textbf{写时复制 (COW)}: 在使用 \textbf{Fork} 创建子进程时,先共享物理页面,
    只有在写操作时才为子进程分配实际物理内存,从而显著提升资源利用率。
\end{itemize}

\datedsubsection{\href{https://github.com/trace1729/rvcpu-emulator}{\textbf{基于 RISC-V 的流水线仿真器}}}{}
\begin{itemize}[leftmargin=*]
    \item \textbf{功能概述}: 使用 C++ 构建单发射、乱序执行的五级流水线模拟器,支持二级 Cache、虚拟内存和简易分支预测。
    \item \textbf{Cache 设计}: 提供可配置的 Non-Inclusive Cache,包括可定制的 Cache Line 大小、写回策略等,以提升缓存性能。
    \item \textbf{虚拟内存}: 使用二级页表实现虚拟内存映射,一级与二级页面索引均为 10 位,提高内存管理效率。
    \item \textbf{乱序执行}: 分别采用 Scoreboard 和 Tomasulo 算法,实现乱序执行与流水线调度,减少指令阻塞并提升吞吐量。
    \item \textbf{可视化调试}: 基于 \textbf{VUE} 构建了可视化调试器,能够实时监控流水线状态和性能指标,便于快速定位瓶颈。
\end{itemize}

\datedsubsection{\href{https://github.com/trace1729/ysyx}{\textbf{一生一芯第六期 B 线}}}{}
\begin{itemize}[leftmargin=*]
    \item \textbf{整体架构}: 基于 Chisel/Verilog 搭建的 RV32IM 流水线处理器,支持异常处理并可运行 RT-Thread 操作系统,
    通过 AXI 总线与外设交互。
    \item \textbf{软件模拟器}: 实现了 RV32E 指令集的行为级模拟器 \textbf{NEMU},
    支持多种 \textbf{Trace} 点与迷你调试器功能,便于观察处理器状态并分析性能瓶颈。
    \item \textbf{差分测试}: 利用开源模拟器 \textbf{spike} 校准 \textbf{NEMU},
    再以 \textbf{NEMU} 为基准校准硬件设计,实现软硬件一致性验证。
\end{itemize}



\section{编程技能}
熟悉 cpp / c、java、python、Android 开发 以及 Linux 系统的使用。

\section{荣誉奖项}
2021-2022学年一等奖学金、优秀共青团员、社会活动积极分子等。


% \section{\faInfo\ 社会实践/其他}
\section{社会活动}
% increase linespacing [parsep=0.5ex]
\begin{itemize}[parsep=0.2ex]
  \item 南昌大学新媒体研发部成员,为部门同学提供 Linux 基础的课程培训。(2021-2023)
  \item 南昌大学百年校庆志愿者 (2021),第八届互联网+创新创新业大赛志愿者(2021)。活动期间,为老师,同学提供技术支持。
\end{itemize}

\end{document}
