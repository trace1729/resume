\documentclass{resume}
\usepackage{zh_CN-Adobefonts_external} 
\usepackage{linespacing_fix}
\usepackage{cite}
\usepackage{fontawesome}
\usepackage{hyperref} 
\begin{document}
\pagenumbering{gobble}



%***"%"后面的所有内容是注释而非代码,不会输出到最后的PDF中
%***使用本模板,只需要参照输出的PDF,在本文档的相应位置做简单替换即可
%***修改之后,输出更新后的PDF,只需要点击Overleaf中的“Recompile”按钮即可
%**********************************姓名********************************************
\name{龚开宸}
%**********************************联系信息****************************************
%第一个括号里写手机号,第二个写邮箱
\contactInfo{18270359936}{gongkch2024@shanghaitech.edu.cn}
\otherInfo{性别:男}{籍贯:江西}{}{}
%**********************************其他信息****************************************
%在大括号内填写其他信息,最多填写4个,但是如果选择不填信息,
%那么大括号必须空着不写,而不能删除大括号。
%\otherInfo后面的四个大括号里的所有信息都会在一行输出
%如果想要写两行,那就用两次这个指令(\otherInfo{}{}{}{})即可
%*********************************照片**********************************************
%照片需要放到images文件夹下,名字必须是you.jpg,如果不需要照片可以不添加此行命令
%0.15的意思是,照片的宽度是页面宽度的0.15倍,调整大小,避免遮挡文字
 \yourphoto{0.12}
%**********************************正文**********************************************


%***大标题,下面有横线做分割
%***一般的标题有:教育背景,实习(项目)经历,工作经历,自我评价,求职意向,等等


%***********一行子标题**************
%***第一个大括号里的内容向左对齐,第二个大括号里的内容向右对齐
%***\textbf{}括号里的字是粗体,\textit{}括号里的字是斜体


%***********列举*********************
%***可添加多个\item,得到多个列举项,类似的也可以用\textbf{}、\textit{}做强调
\section{基本信息}
\textbf{本科 GPA: } 3.72 / 4 \\
\textbf{本科 专业排名: } 12 / 211 (5.3\%)  \\
\textbf{语言技能: } CET-4: 571 \quad CET-6: 568 \\
\begin{tabular}{@{}p{0.7\textwidth}p{0.3\textwidth}@{}}
  \textbf{本科就读院校: } 南昌大学 计算机科学与技术(\textbf{211工程, 双一流建设高校}) & \hfill \textbf{2020-2024} \\
  \textbf{研究生在读院校: } 上海科技大学 计算机科学与技术(\textbf{双一流建设高校}) & \hfill 2024-2027
\end{tabular}

% \end{itemize}


\section{项目经历}

\datedsubsection{\href{https://git.acwing.com/trace4869/xv6}{ \textbf{XV6 操作系统内核改进} }}{}
  XV6 是一个类 unix 操作系统。在阅读源码和官方手册的基础上,为XV6增加了新功能,优化了 XV6 的性能。在项目中加深了对操作系统的理
  解,锻炼了系统编程的能力。
 \\   \textbf{功能实现}:  
\begin{itemize}
 \item 为 XV6 增加了 backtrace, sysinfo 等系统调用。backtrace 可以根据栈帧打印函数调用链,能够提升调试
 的效率。sysinfo 能在终端打印出系统目前所可用的内存大小和目前非阻塞状态的进程数量, 能有效的监控系统运行情况。
 \item 重新设计了 XV6 的内存分配器。为每一个进程单独分配空闲内存链表,降低了 kalloc 实现中锁的竞争,提高了内存分配的效率。
 \item 为XV6进行了写时复制(copy on write)。 在使用 fork 创建子进程时,不分配新空间,只拷贝引用。当子进程试图修改资源内容时,再
 为其分配物理内存。
\end{itemize}

\datedsubsection{\href{https://github.com/trace1729/ysyx}{ \textbf{ 一生一芯第六期B线 } }}{}
  基于 Chisel/Verilog 的流水线 RISC-V 处理器。 
 \\   \textbf{主要内容}:  
  \begin{itemize}
    \item \textbf{软件} 实现RV32E行为模拟器NEMU (NJU-emulator), 为处理器添加各类 trace, 以及一个迷你调试器。
    \item \textbf{硬件} 使用 chisel/verilog 实现的顺序5级流水线架构的RV32IM处理器, 支持异常能够运行RT-Thread系统, 可以通过AXI总线
    连接到外设。
    \item \textbf{测试} 本项目采取差分测试的方式验证处理器的行为,先通过开源模拟器spike校准NEMU,再通过NEMU校准处理器。
\end{itemize}

\datedsubsection{\href{https://github.com/trace1729/rvcpu-emulator}{ \textbf{ 基于RISC-V的流水线仿真器 } }}{}
 使用 C++ 实现了支持二级Cache、虚拟内存、简单分支预测的乱序单发射五级流水线模拟器。
 \\   \textbf{主要内容}:  
  \begin{itemize}
    \item \textbf{Cache 配置} 可配置的Non-inclusive Cache,可通过配置文件指定Cache Line的大小 以及 Cache 的写回策略等。 
    \item \textbf{虚拟内存 配置} 使用二级页表实现虚拟内存,一级、二级页面数都是10位。
    \item \textbf{乱序算法} 分别使用Scoreboard算法和tomasulo算法实现处理器的乱序。
    \item \textbf{项目特色} 基于 vue-press 可视化调试器。
\end{itemize}


\section{编程技能}
熟悉 cpp / c、java、python、Android 开发 以及 Linux 系统的使用。

\section{荣誉奖项}
2021-2022学年一等奖学金、优秀共青团员、社会活动积极分子等。


% \section{\faInfo\ 社会实践/其他}
\section{社会活动}
% increase linespacing [parsep=0.5ex]
\begin{itemize}[parsep=0.2ex]
  \item 南昌大学新媒体研发部成员,为部门同学提供 Linux 基础的课程培训。(2021-2023)
  \item 南昌大学百年校庆志愿者 (2021),第八届互联网+创新创新业大赛志愿者(2021)。活动期间,为老师,同学提供技术支持。
\end{itemize}

\end{document}
