\documentclass{resume}
\usepackage{zh_CN-Adobefonts_external} 
\usepackage{linespacing_fix}
\usepackage{cite}
\usepackage{fontawesome}
\usepackage{hyperref} 
\begin{document}
\pagenumbering{gobble}



%***"%"后面的所有内容是注释而非代码,不会输出到最后的PDF中
%***使用本模板,只需要参照输出的PDF,在本文档的相应位置做简单替换即可
%***修改之后,输出更新后的PDF,只需要点击Overleaf中的“Recompile”按钮即可
%**********************************姓名********************************************
\name{龚开宸}
%**********************************联系信息****************************************
%第一个括号里写手机号,第二个写邮箱
\contactInfo{18270359936}{kaichen@email.ncu.edu.cn}
\otherInfo{性别:男}{籍贯:江西}{}{}
%**********************************其他信息****************************************
%在大括号内填写其他信息,最多填写4个,但是如果选择不填信息,
%那么大括号必须空着不写,而不能删除大括号。
%\otherInfo后面的四个大括号里的所有信息都会在一行输出
%如果想要写两行,那就用两次这个指令(\otherInfo{}{}{}{})即可
%*********************************照片**********************************************
%照片需要放到images文件夹下,名字必须是you.jpg,如果不需要照片可以不添加此行命令
%0.15的意思是,照片的宽度是页面宽度的0.15倍,调整大小,避免遮挡文字
 \yourphoto{0.15}
%**********************************正文**********************************************


%***大标题,下面有横线做分割
%***一般的标题有:教育背景,实习(项目)经历,工作经历,自我评价,求职意向,等等


%***********一行子标题**************
%***第一个大括号里的内容向左对齐,第二个大括号里的内容向右对齐
%***\textbf{}括号里的字是粗体,\textit{}括号里的字是斜体


%***********列举*********************
%***可添加多个\item,得到多个列举项,类似的也可以用\textbf{}、\textit{}做强调
\section{基本信息}
\textbf{GPA: } 3.72 / 4 \\
\textbf{专业排名: } 12 / 211 (5.3\%)  \\
\textbf{语言技能: } CET-4: 571 \quad CET-6: 568 \\
\textbf{本科院校: } 南昌大学-计算机科学与技术(211工程, 双一流建设高校) 

% \end{itemize}


\section{项目经历}

\datedsubsection{\href{https://git.acwing.com/trace4869/xv6}{ \textbf{XV6 操作系统内核改进} }}{}
  XV6 是一个类 unix 操作系统。在阅读源码和官方手册的基础上,为XV6增加了新功能,优化了 XV6 的性能。在项目中加深了对操作系统的理
  解,锻炼了系统编程的能力。
 \\   \textbf{功能实现}:  
\begin{itemize}
 \item 为 XV6 增加了 backtrace, sysinfo 等系统调用。backtrace 可以根据栈帧打印函数调用链,能够提升调试
 的效率。sysinfo 能在终端打印出系统目前所可用的内存大小和目前非阻塞状态的进程数量, 能有效的监控系统运行情况。
 \item 为XV6添加一个用户中断,可通过系统调用设置周期性的执行用户指定的handler函数。
 这个功能可以为一些计算密集型程序的设计调试提供帮助。
 \item 为XV6进行了写时复制(copy on write)。 在使用 fork 创建子进程时,不分配新空间,只拷贝引用。当子进程试图修改资源内容时,再
 为其分配物理内存。
\end{itemize}

\datedsubsection{\href{https://git.acwing.com/trace4869/riscv-cpu}{ \textbf{ 基于 Logisim 的 RISC-V CPU} }}{}
  使用 Logisim 实现了二级流水的 RISC-V CPU。支持 \textbf{R}, \textbf{I}, \textbf{S}, 
    \textbf{B}, \textbf{U}, \textbf{J} 六种类型的指令,共 35 条。 
 \\   \textbf{实现过程}:  
  \begin{itemize}
    \item \textbf{模块化设计 } 将CPU分为5个子模块: ALU、 RegFile、  
     Imm-Gen、 Data Path 和 Control Logic。
    \item \textbf{指令实现 } 先完成 ALU、 RegFile 模块,之后从 I 型指令开始,不断完善Data Path、 Control Logic 和 Imm-Gen。
    完成一类指令后,对 CPU 进行集成测试。
    \item \textbf{流水线设计 } 将 CPU 的运行过程分为两个阶段,取指令(IF)和指令执行(EX)。在两个
    阶段之间,通过寄存器保存程序计数器(PC)和PC对应的指令(INS),
    此外,还需通过控制信号PCsel和多路选择器来解决因条件转移指令带来的控制冒险。 

\end{itemize}

% \datedsubsection{\href{https://gitee.com/trace1729/dl}{ \textbf{基于 Android Compose 的音乐播放器 } }}{}
%   使用声明式编程开发的一款音乐播放器,采用了 MVVM 架构。 
%  \\   \textbf{实现过程}:  
% \begin{itemize}
%   \item \textbf{界面导航: } 采用 Navigation 组件进行页面导航,并且将播放器的全部状态提升到 Navhost, 再由 Navhost 将
%   状态传递给页面
%   \item \textbf{音乐播放: } 首先将python的爬取服务部署在服务器上,播放器可向服务器发送网络请求,
%     爬虫服务接受到网络请求后爬取数据,并将获得的数据返回给播放器。播放器通过 ExoPlayer 这个组件解析获取到的音乐数据,
%     进行播放。
%   \item \textbf{数据持久化: } 音乐播放器支持登入、注册与储存用户偏好。通过 Room 组件,将用户的账号和加密后的
%   密码存储手机上的 sqlite 中;通过 Datastore 组件将用户偏好储存在文件中。
% \end{itemize}

\datedsubsection{\href{https://gitee.com/trace1729/dl}{ \textbf{基于 经典卷积神经网络的图片分类} }}{}
阅读经典神经网络论文,并尝试复现。
\begin{itemize}
  \item 数据集 CIFAR-10,训练框架Pytorch,训练平台NVIDA 3070 (内存8G)。
  \item 阅读了经典神经网络 AlexNet、VGG、GoogLeNet 以及 ResNet 的相关论文。对论文中采用的训练方案进行了复现
  此外还了解 batch normalization、kaiming 初始化等相关论文工作。
  \item \textbf{结论} VGG 网络占用内存最大(batch size 为 128 下,内存占用为7648MB);采用预训练的ResNet分类效果最好,
  正确率为 98.3\%。 
\end{itemize} 

% \section{主干课程}
% 高等数学: \textbf{93 }  
% 数字逻辑:  \textbf{90  }
% 数值计算:  \textbf{90  }
% 面向对象程序设计:  \textbf{95 }
% 数据库原理:  \textbf{93}
% 数据结构实践:  \textbf{97}

\section{编程技能}
熟悉 cpp / c、java、python、Android 开发 以及 Linux 系统的使用。

\section{荣誉奖项}
2021-2022学年一等奖学金、优秀共青团员、社会活动积极分子等。


% \section{\faInfo\ 社会实践/其他}
\section{社会活动}
% increase linespacing [parsep=0.5ex]
\begin{itemize}[parsep=0.2ex]
  \item 南昌大学新媒体研发部成员,为部门同学提供 Linux 基础的课程培训。(2021-2023)
  \item 南昌大学百年校庆志愿者 (2021),第八届互联网+创新创新业大赛志愿者(2021)。活动期间,为老师,同学提供技术支持。
\end{itemize}

\end{document}
