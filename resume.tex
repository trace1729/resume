\documentclass{resume}
\usepackage{zh_CN-Adobefonts_external} 
\usepackage{linespacing_fix}
\usepackage{cite}
\usepackage{fontawesome}
\usepackage{hyperref} 
\begin{document}
\pagenumbering{gobble}



%***"%"后面的所有内容是注释而非代码,不会输出到最后的PDF中
%***使用本模板,只需要参照输出的PDF,在本文档的相应位置做简单替换即可
%***修改之后,输出更新后的PDF,只需要点击Overleaf中的“Recompile”按钮即可
%**********************************姓名********************************************
\name{龚开宸}
%**********************************联系信息****************************************
%第一个括号里写手机号,第二个写邮箱
\contactInfo{18270359936}{trace1729@gmail.com}
\otherInfo{性别:男}{籍贯:江西}{}{}
%**********************************其他信息****************************************
%在大括号内填写其他信息,最多填写4个,但是如果选择不填信息,
%那么大括号必须空着不写,而不能删除大括号。
%\otherInfo后面的四个大括号里的所有信息都会在一行输出
%如果想要写两行,那就用两次这个指令(\otherInfo{}{}{}{})即可
%*********************************照片**********************************************
%照片需要放到images文件夹下,名字必须是you.jpg,如果不需要照片可以不添加此行命令
%0.15的意思是,照片的宽度是页面宽度的0.15倍,调整大小,避免遮挡文字
 \yourphoto{0.15}
%**********************************正文**********************************************


%***大标题,下面有横线做分割
%***一般的标题有:教育背景,实习(项目)经历,工作经历,自我评价,求职意向,等等


%***********一行子标题**************
%***第一个大括号里的内容向左对齐,第二个大括号里的内容向右对齐
%***\textbf{}括号里的字是粗体,\textit{}括号里的字是斜体


%***********列举*********************
%***可添加多个\item,得到多个列举项,类似的也可以用\textbf{}、\textit{}做强调
\section{基本信息}
\textbf{GPA: } 3.72 / 4 \\
\textbf{专业排名: } 12 / 211 (5.3\%)  \\
\textbf{语言技能: } CET-4: 571 \quad CET-6: 568 \\
\textbf{本科院校: } 南昌大学-计算机科学与技术(211工程, 双一流建设高校) 

% \end{itemize}

\section{技术能力}
\begin{itemize}
  \item \textbf{编程语言: } 熟悉 c/cpp, java, python, kotlin, shell
\end{itemize}

\section{项目经历}
\datedsubsection{\href{https://git.acwing.com/trace4869/riscv-cpu}{ \textbf{ 基于 Logisim 的 RISC-V CPU} }}{}
  使用 Logisim 实现了二级流水的 RISC-V CPU。支持 \textbf{R}, \textbf{I}, \textbf{S}, 
    \textbf{B}, \textbf{U}, \textbf{J} 六种类型的指令,共 35 条。 
 \\   \textbf{实现过程}:  
  \begin{itemize}
    \item 通过 Logisim 的隧道功能,可以将CPU分为6个子模块: ALU, RegFile,  
    Save \& Load, Imm-Gen, Data Path, Control Logic。
    \item 由于前三个模块功能相对独立,在 Logisim 实现后,可对其进行单元测试。前三个模块构建完成后,就可以开始搭建数据通路(Data Path),控制逻辑(Control Logic) 和 立即数生成器(Imm-Gen)。
    由于不同类型的指令,对应的后三个模块的结构也相对独立。我从最简单的\textbf{I} 型指令开始构建。完成一类指令结构的设计后,就可以
    进行集成测试。 
    \item 实现所有指令类型后,即可进行流水线设计。我将 CPU 的运行过程分为两个阶段,取指令(IF)和指令执行(EX)。在两个
    阶段之间,通过寄存器保存程序计数器(PC)和PC对应的指令(INS),这样在 IF 和 EX 两个阶段有着不同的 PC, INS。此外,还需
    解决因条件转移指令带来的控制冒险。我们可以从控制信号 PCsel 知道是否需要跳转,需要跳转的话就插入一条 无操作指令 nop 来消除冒险。 

\end{itemize}

\datedsubsection{\href{https://git.acwing.com/trace4869/xv6}{ \textbf{XV6 操作系统内核优化} }}{}
  XV6 是一个类 unix 操作系统。在阅读源码和官方手册的基础上,为XV6增加了新功能,优化了 XV6 的性能。在项目中加深了对操作系统的理
  解,锻炼了系统编程的能力。
 \\   \textbf{功能实现}:  
\begin{itemize}
 \item 为 XV6 增加了 backtrace, sysinfo 等系统调用。backtrace 可以追踪一个系统调用函数在运行过程中的调用情况,能够提升调试
 的效率。sysinfo 能在终端打印出系统目前所可用的内存大小和正在运行的CPU核心数量, 能有效的监控系统运行情况。
 \item 实现了页表的访问检测,通过给定起始和终止地址, XV6可以找出这段地址空间上访问过的页表。页表的访问信息可以为
 垃圾回收器提供一个良好的参考,提升系统运行的效率。
 \item 为XV6添加一个时钟中断。为一个进程设置好中断后,每隔n个时钟周期,XV6会自动中断运行进程,在终端打进程占用资源情况,
 再恢复之前进程的执行。这个功能可以为一些计算密集型程序的设计调试提供帮助。
\end{itemize}

\datedsubsection{\href{https://gitee.com/trace1729/audio-app}{ \textbf{基于 Android Compose 的音乐播放器 } }}{}
  使用声明式编程开发的一款音乐播放器,采用了 MVVM 架构。 
 \\   \textbf{实现过程}:  
\begin{itemize}
 %  \item 在阅读了 Google 关于 Android 的官方文档后,将声明式开放与命令式开放进行了对比,最终选择了命令式开发。
  \item \textbf{界面导航: } 采用 Navigation 组件进行页面导航,并且将播放器的全部状态提升到 Navhost, 再由 Navhost 将
  状态传递给页面
  \item \textbf{音乐播放: } 首先将python的爬取服务部署在服务器上,播放器可向服务器发送网络请求,
    爬虫服务接受到网络请求后爬取数据,并将获得的数据返回给播放器。播放器通过 ExoPlayer 这个组件解析获取到的音乐数据,
    进行播放。
  \item \textbf{数据持久化: } 音乐播放器支持登入、注册与储存用户偏好。通过 Room 组件,将用户的账号和加密后的
  密码存储手机上的 sqlite 中;通过 Datastore 组件将用户偏好储存在文件中。
  \item \textbf{总结: } 采用声明式编程进行开发,代码的可读性会更高;且程序可以在低层次进行代码复用,有效的提升了程序运行的性能。
\end{itemize}

\section{荣誉奖项}
2021-2022学年一等奖学金、优秀共青团员、社会活动积极分子等。

% \section{\faInfo\ 社会实践/其他}
\section{社区参与/实践其他}
% increase linespacing [parsep=0.5ex]
\begin{itemize}[parsep=0.2ex]
  \item 南昌大学新媒体研发部成员,为部门同学提供 Linux 基础的课程培训。(2021-2023)
  \item 南昌大学百年校庆志愿者 (2021),第八届互联网+创新创新业大赛志愿者(2022)。活动期间,为老师,同学提供技术支持。
\end{itemize}

\end{document}
